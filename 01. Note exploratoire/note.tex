\documentclass{article}
\usepackage[utf8]{inputenc}

\title{Note exploratoire}
\author{mael.chaumette }
\date{October 2021}

\begin{document}

\maketitle

\section{Partie 1}
\subsection{Définition des termes du sujet}
\subsubsection{territoire dynamique}
Changement, évolution et, par extension, capacité à changer, à évoluer. Au demeurant, la notion ne doit pas être interprétée uniquement en termes de croissance positive. Une dynamique, dans telle situation socio-spatiale, peut-être négative, elle peut traduire le déclin, la déshérence, la déprise. La dynamique des territoires étudie les changements qui sont en œuvre du point de vue :
\begin{itemize}
    \item des localisations des populations et de leurs activités,
    \item des aménagements et des capacités de maîtrise des territoires étudiés.
\end{itemize}
On pourra analyser différents types de dynamiques spatiales avec leurs manifestations : fronts pionniers, mutations territoriales (urbaines, rurales), dynamiques de la mondialisation, etc.

\subsubsection{territoire attractif}
Capacité à attirer dans une direction, vers un lieu ou vers une aire, l'attractivité est centripète et cumulative. Elle est à la source des concentrations, des polarisations et de divers phénomènes intéressant le géographe : la métropolisation, les migrations par exemple.
La répulsion, à l'opposé de l'attraction, est centrifuge. Attraction et répulsion sont sources de mobilités et peuvent obéir à des logiques gravitaires.

\subsection{Synthèse}

démographie, population

\medbreak

économie, emploi


\section{Partie 2}



\end{document}
