\documentclass{article}
\usepackage[utf8]{inputenc}
\usepackage[T1]{fontenc}
\usepackage[french]{babel}
\usepackage{multirow}
\usepackage{pdfpages}
\usepackage{geometry}
\usepackage{enumitem}
\usepackage{hyperref}
\geometry{hmargin=2.5cm,vmargin=1.5cm}
 \frenchbsetup{StandardLists=true}

\title{Dynamisme et attractivité des territoires \\ Note exploratoire}
\author{Maël Chaumette, Ludovic Deneuville, Thomas Soulié}
\date{Octobre 2021}

\begin{document}

\maketitle

\section{Définition des termes du sujet}

La première étape de notre analyse consiste à définir les termes du sujet. Le site internet geoconfluences.ens-lyon.fr  propose dans son glossaire des définitions intéressantes sur le dynamisme et l'attractivité. 
\par
Le dynamisme est défini comme «~Changement, évolution et, par extension, capacité à changer, à évoluer~» qui peut aussi bien être un changement positif ou négatif. Cependant lorsque le qualificatif «~dynamique~» est employé, c'est souvent avec une connotation positive. La dynamique des territoires étudie les changements, principalement les mouvements de population. 

\bigbreak

\par
L'attractivité est la «~Capacité à attirer dans une direction, vers un lieu ou vers une aire~». Elle a tendance à former un cycle vertueux, c'est à dire qu'un territoire attractif est souvent de plus en plus attractif et inversement. Pour un territoire en déficit d’attractivité, il est difficile de modifier cette dynamique sans prendre des initiatives fortes.
Le terme attractivité semble également lié à l'adjectif économique, comme cela sera décrit dans les études présentées ci-dessous. 




\section{Références bibliographiques autour du thème}

Le dynamisme d'un territoire peut se mesurer en suivant les évolutions démographiques, c'est-à-dire le taux d'accroissement naturel et le flux migratoire. Les documents choisissent des échelles différentes : 
régionale~[1][2], intercommunalité~[3], départementale~[4], zone d'emploi~[5]. 

\bigbreak

De même, l'attractivité d'un territoire contribue à son dynamisme. Cependant, l'attractivité dépend de plusieurs paramètres : le tourisme, l'enclavement, la concurrence avec d'autres territoires~[6], la qualité de vie~[2], ce que la note~[7] nous résume sous la forme des capacités d'être et d'agir. L'ensemble des articles se complètent et montrent que les populations ont tendance à se déplacer vers les zones urbaines. Les articles~[3][4][5] montrent une importance du temps, en effet l'attractivité d'un territoire évolue au cours du temps et montre de grandes disparités entre les territoires. Nous pouvons voir que certaines zones urbaines sont délaissées au profit d'autres. Dans l'étude~[3], 46\% des personnes qui se sont installées dans la région Bourgogne-Franche-Comté en 2016 était originaire d'une autre région de France. 

\bigbreak

De plus, les populations se déplacent aussi vers les littoraux. L'article~[1] dégage ainsi une tendance en forme de «~U~», à l'échelle de la France, pour les territoires ayant les meilleurs taux de variation de la population. Les documents s'accordent sur le fait que le nord de la France ainsi que le centre (à l'exception de Paris) sont moins attractifs et ont tendance à décliner au profit du sud. De la même manière, les zones rurales sont en perte de vitesse au profit des aires urbaines. 

\bigbreak

L'étude nationale~[8] traite de l'attractivité économique des territoires et cite différents aspects de l'attractivité économique. Elle décrit via des exemples les spécificités de certains territoires ainsi que leurs forces et leurs faiblesses. Le découpage utilisé est la zone d'emploi ce qui permet d'avoir une vision fine. Les zones d'emplois, au nombre de 304 en France métropolitaine sont des zones géographiques où la plupart des actifs résident et travaillent. La principale caractéristique d'un territoire attractif économiquement serait un territoire qui attire des emplois. Cependant, ce n'est pas le seul critère. Pour devenir attractif, un territoire peut miser également sur la venue de touriste, la venue de retraités, son offre de loisirs, son offre de transports ou un cadre de vie agréable. Tous les territoires n'ont pas les mêmes atouts, certains sont proches d'une grande métropole, d'autres ont l'avantage d'être littoraux ou montagneux ou encore ont un patrimoine culturel important. Dans cette étude, l'attractivité économique est divisée en deux parties :

\begin{itemize}[label=\textbullet]
    \item l'attractivité économique productive, c'est à dire l'implantation de nouvelles activités, la création d'emplois, idéalement des emplois qualifiés qui ont un effet catalyseur sur l'économie locale, 
    \item l'attractivité économique résidentielle, où l'objectif est d'attirer des «~revenus~», que ce soit les touristes, les retraités ou les navetteurs (salariés qui travaillent en dehors du territoire). Leurs dépenses permettent de soutenir l'économie locale.
\end{itemize}

\noindent Les conclusions de cette étude sont que la plupart des territoires arrivent à tirer leur épingle du jeu, à l'exception du grand sud-ouest qui attire peu de navetteurs et du nord-est. 

\bigbreak

Une étude régionale~[9] sur l'attractivité économique du Centre-Val de Loire présente une région qui se situe dans la moyenne nationale pour la plupart des indicateurs. Cette étude utilise les mêmes éléments que la précédente, elle traite un à un les différents aspects de l'attractivité au niveau des territoires de la région, avec comme atouts principaux la proximité de Paris qui attire de nombreux navetteurs et la richesse culturelle des châteaux de la Loire pour le tourisme.

\bigbreak

Une autre étude~[10] s'est déroulée à un niveau plus ciblé, sur la dynamique du grand Narbonne. La méthode utilisée pour évaluer les critères de dynamisme et d’attractivité est de comparer les résultats de la zone étudiée par rapport à d'autres communautés d'agglomération présentant des caractéristiques proches, en terme de population, de secteur des emplois et de présence de littoral. Cela permet de situer le grand Narbonne et de noter que l'emploi et le parc de logement sont en croissance ainsi le niveau d'équipement est très satisfaisant. Parmi les points négatifs, ressortent un nombre important de personnes en situation de précarité et que le territoire est vieillissant.

\section*{Bibliographie}

\noindent
[1] \href{https://www.insee.fr/fr/statistiques/4999744}{INSEE Focus n°221 - Le dynamisme démographique faiblit entre 2013 et 2018, avec la dégradation du solde naturel, dec. 2020}
\\
\noindent
[2] \href{https://www.insee.fr/fr/statistiques/3545995}{INSEE Document de travail n°H2018/02 - Les dynamiques de la qualité de vie dans les territoires, mai 2018}
\\
\noindent
[3] \href{https://www.insee.fr/fr/statistiques/4768297}{INSEE Analyse Bourgogne-Franche-Comté n°75 - Attractivité résidentielle : des ressorts pas toujours suffisants dans les grands pôles urbains, des atouts dans certaines petites centralités, oct. 2020}
\\
\noindent
[4] \href{https://www.insee.fr/fr/statistiques/4806684}{INSEE Focus n°210 - Toujours plus d’habitants dans les unités urbaines, oct. 2020}
\\
\noindent
[5] \href{https://www.insee.fr/fr/statistiques/2411498}{INSEE Première n°1622 - Arrivées d'emplois et de résidents - Un enjeu pour les territoires, nov. 2016}
\\
\noindent
[6] \href{http://www.adeus.org/productions/attractivites-des-territoires-du-grand-est-liens-emplois-logements/files/attractivites_territoires_synthese_action_logement.pdf}{Adeus : Attractivités des territoires : liens emplois/logements, jan. 2020} 
\\
\noindent
[7] \href{https://www.cairn.info/revue-mondes-en-developpement-2010-1-page-27.htm\#s2n2}{CAIRN : L'attractivité des territoires : un concept multidimensionnel, Jacques Poirot, Hubert Gérardin, jan. 2010}
\\
\noindent
[8] \href{https://www.insee.fr/fr/statistiques/1281062}{INSEE Première n°1416 - L'attractivité économique des territoires,  oct. 2012}
\\
\noindent
[9] \href{https://www.insee.fr/fr/statistiques/2422226}{INSEE Analyses Centre-Val de Loire n°28 - Une attractivité économique des territoires à soutenir, nov. 2016}
\\
\noindent
 [10] \href{https://www.insee.fr/fr/statistiques/2663749}{INSEE Analyses Occitanie n°41 - Le Grand Narbonne Un territoire dynamique porté par son attractivité, mars 2017}

\newpage

\section{Sélection des variables d'intérêt}

Pour la sélection des différentes variables à exploiter, nous nous sommes demandés quels étaient les variables qui peuvent le mieux représenter le dynamisme et/ou l'attractivité d'un territoire. Notre choix s'est porté sur les variables ci-dessous :

\begin{itemize}[label=\textbullet]
    \item Emploi
    \begin{itemize}[label=$\circ$]
        \item le taux de chômage des 15 ans et plus, 
        \item le taux d'activité des 15-24 ans,
        \item la part des "retraités" dans la population,
        \item la part des "cadres et professions intellectuelles supérieures" dans la population,
        \item la part des ouvriers dans la population,
        \item le ratio entre les "cadres" et les "ouvriers",
    \end{itemize}
    \item Démographie
    \begin{itemize}[label=$\circ$]
        \item le nombre de logements commencés pour 1000 habitants, 
        \item la valeur ajoutée régionale totale, 
        \item le taux d'évolution annuel de l'emploi, 
        \item le taux d'évolution annuel de la population due au solde migratoire, 
        \item le taux d'évolution annuel de la population due au solde naturel, 
        \item le nombre d'équipements sportifs et culturels de la gamme supérieure, 
        \item le taux d'évolution annuel de la population, 
    \end{itemize}
    \item Mobilité
    \begin{itemize}[label=$\circ$]
        \item la part des ménages ayant emménagé depuis moins de 2 ans,
        \item la part des actifs occupés résidant à 30 minutes ou moins de leur lieu de travail,
        \item le nombre de nuitées dans des hébergements touristiques liées au tourisme national.
    \end{itemize}
\end{itemize}

\medbreak

Avec SAS et la procédure \textbf{MEANS}, nous obtenons les valeurs de différents indicateurs pour la variable du taux d'évolution annuel de l'emploi. Nous avons résumé ces valeurs dans le tableau ci-dessous : 

\bigbreak

\begin{center}
    
\begin{tabular}{|c|c|r|}
    \hline
     & Indicateurs & Résultats \\
     \hline
     \multirow{4}{*}{Taux d'évolution annuel de l'emploi} & Moyenne & -0.14\\
     & Médiane & -0.16 \\
     & Variance & 1.77 \\
     & Ecart-type & 1.33 \\
     \hline
\end{tabular}

\end{center}

\bigbreak

De plus, avec la procédure \textbf{SGPLOT}, nous avons obtenu la boîte à moustache du taux d'évolution annuel de l'emploi : 

\begin{figure}[h]
    \centering
    \includegraphics[scale=0.4]{BàM.png}
    \caption{Boîte à moustache appliquée au taux d'évolution annuel de l'emploi}
    \label{fig:my_label}
\end{figure}

\end{document}
